\section{Introduction}
Having a precise prediction for the wind velocity in very localized zones has become a relevant necessity for some productive activities like: wind resource assessment,  dispersion control of gases and toxic particulate material and also natural disasters like wildfires and tornadoes.
%El poseer una correcta predicción de la velocidad del viento en ciertas zonas muy localizadas se ha convertido en una necesidad relevante para ciertas actividades productivas como por ejemplo: la estimación del recurso eólico, el control de la dispersión de gases o material particulado contaminante e incluso para el caso de desastres naturales como incendios o tornados.

Historically two ways has been taken to predict wind behavior. The first is an statistical approach that uses data from several meteorological masts or other instrumentation located in a domain to extrapolate information to a time of interest. This approach present some weakness: (i) dependency of the instrumentation, (ii) the historical databases can't capture actual and local conditions of the atmosphere such as climate change, and (iii) the statistical values don't show the behavior of the continuous terrain but in some arbitrary points. Is because this that for more specific goals a second approach is used: the Numerical Weather Prediction (NWP).
%Históricamente se han tomado dos caminos para predecir el comportamiento del viento. El primero, es un enfoque estadístico que utiliza bases de datos de distintos mástiles meteorológicos u otros instrumentos de medición ubicados en un dominio para extrapolar información a un tiempo de interés. Este enfoque presenta ciertas debilidades: (i) la dependencia de la instrumentación, (ii) las bases de datos históricas no permiten captar condiciones actuales y locales como el cambio climático y, (iii) los valores estadísticos no muestran el comportamiento real de todo el continuo del terreno si no que en ciertos puntos arbitrarios. Es por esto que para fines mas específicos se utiliza un segundo acercamiento que es la predicción numérica del clima (NWP según sus siglas en inglés).

The NWP target is to find the future state of the meteorological variables integrating the system of partial differential equations that models the atmosphere behavior. One of the most relevant property of this system is the presence of multiples scales, i.e. the preponderant forces that governs the air dynamics vary depending of the space-temporal scale to analyze. Is convenient then to separate the spatial dependence associating a characteristic length to the domain of interest, in this way a global scale, synoptic scale, mesoscale and microscale can be defined.
%La predicción numérica del clima tiene por objetivo encontrar el estado futuro de las variables meteorológicas integrando el sistema de ecuaciones diferenciales parciales que modelan el comportamiento de la atmósfera. Dentro de las características mas relevantes del problema de resolver este sistema, está la presencia de múltiples escalas, es decir, las fuerzas preponderantes que controlan el movimiento del aire varían dependiendo de la escala espacio-temporal a analizar. Es conveniente separar entonces la dependencia espacial asociando un largo característico al dominio de interés, de esta manera se puede hablar de una escala global, escala sinóptica, mesoescala o microescala. 

The scales multiplicity introduces a new challenge to overcome: the computational cost of solving a system of equations valid for the entire atmosphere. As a consequence of this, a spectrum of numerical models have been created specifically for each spatial scale with their own equations, but the initialization of a small-scale model requires the results of a larger one.
%La multiplicidad de escalas introduce un nuevo desafió: superar el costo computacional que significa resolver un sistema de ecuaciones válido para toda la atmósfera. Como consecuencia de esto se han creado una gama de modelos numéricos diseñados específicamente para cada escala espacial, con sus propias ecuaciones, sin embargo la inicialización de un modelo de escala pequeña requiere los resultados de otro mas grande.

With respect to global models, these have shown to be able to correctly simulate many aspects of the general circulation of the atmosphere \citep{stocker2013climate}, however, for engineering interests, the focus is on the local behavior of the wind, specifically, how it moves within the planetary boundary layer (PBL) which is the part of the atmosphere we inhabit and that is outside the resolution of global models.
%Con respecto a los modelos globales, estos han mostrado ser capaz de simular correctamente muchos aspectos de la circulación general de la atmósfera \citep{stocker2013climate}, sin embargo, para intereses ingenieriles, el foco está en el comportamiento local del viento, en específico, como se mueve dentro de la capa límite planetaria (CLP) que es la parte de la atmósfera en donde habitamos y que queda fuera de la resolución de los modelos globales.

The approach being used today for small-scale atmospheric simulation, i.e. solving the structures belonging to the PBL, is through the so-called dynamic downscaling, interpolating the results from a large-scale model to a small-scale model in order to function as a boundary condition and generate a forecast in a finer mesh. This method defines what is understood by multi-scale simulation and this type of simulation is still widely discussed by the scientific community \citep{Arnold2010}.
%La manera en la que el día de hoy se esta llevando a cabo la simulación atmosférica a pequeñas escalas, i.e. resolviendo las estructuras pertenecientes a la CLP, es a través de la técnica de escalamiento dinámico (\emph{dynamic downscaling}), interpolando los resultados de un modelo de gran escala a otro de pequeña escala, para así funcionar como condición de borde y generar un pronóstico en una malla mas fina. Este acercamiento define lo se que entiende por simulación multiescala y este tipo de simulaciones aún son ampliamente discutidas por la comunidad científica \citep{Arnold2010}.

The use of dynamic downscaling proved to be successful at least in the spectrum of global and synoptic scales. Numerical issues arise in the mesoscale due to terrain forcing and the relevance of local surface fluxes. The reduction to the microscale causes the increase in the relevance of turbulent stresses in the equations, requiring a much more precise handling.
%La utilización del escalamiento dinámico ha tenido buenos resultados por lo menos en el espectro de escalas globales y sinópticas. En la mesoescala surgen complicaciones numéricas debido a los forzamientos por la forma del terreno y que se discutirán en la segunda parte de este artículo. Por otra parte la reducción hasta la microescala causa el aumento en la relevancia de los esfuerzos turbulentos en la ecuaciones, exigiendo un tratamiento mucho mas delicado.

%La turbulencia en la atmósfera se puede manifestar de distintas formas: (a) turbulencia en la atmósfera libre (aquella que afecta a los aviones), (b) turbulencia asociada al roce con el suelo, (c) turbulencia asociada a la flotación por efectos térmicos y (d) turbulencia asociada a la interacción con obstáculos. Estas últimas tres son fenómenos específicos de la CLP. En las grandes escalas la turbulencia no es causa fundamental de los movimientos de la atmósfera, sino que su importancia está limitada en transmitir la información de lo que ocurre en la superficie terrestre.

Due to the space-time numerical grid dimensions in large scale models, the turbulence associated with the interaction with the surface and the thermal effects are generally parameterized through a turbulent viscosity. In scales close to the microscale, the models start natively to solve the turbulent structures generating a problem due to the double weighting of these structures as they are being solved on the one hand and parameterized on the other. This zone is known as the grey zone \citep{Wyngaard2004} and an incorrect configuration of the dynamic downscaling in this zone can cause non-physical results of the model. At the microscale it is possible to represent the turbulence according to known numerical models such as LES or RANS depending on the case.
%Debido a las dimensiones de la malla numérica espacio-temporal de los modelos de gran escala, la turbulencia asociada a la interacción con el suelo y a los efectos térmicos quedan, generalmente, parametrizados a través de una viscosidad turbulenta. En  las escalas cercanas a la microescala, los modelos comienzan nativamente a resolver las estructuras turbulentas generando un problema debido a la doble ponderación de estas ya que por una parte están siendo resueltas y por otra parametrizadas. A esta zona se le conoce como zona gris \citep{Wyngaard2004} y una incorrecta configuración del escalamiento dinámico en esta zona puede ocasionar resultados no físicos del modelo. Ya en la microescala es posible representar la turbulencia según modelos numéricos conocidos como LES o RANS según cual sea el caso.

Atmospheric models such as WRF have been widely used in recent years to predict wind behavior at the mesoscale through multiscale simulations, but only a few studies have addressed this behavior at the microscale in real simulations. The researches that analyze the behavior of the LES to represent the PBL generally use ideal conditions (e.g. periodic boundary conditions, flat terrain, imposed pressure gradients) which allows the validation of the approach, but at the cost of losing the operativeness of working in a realistic scenario. Simulating a real case implies the use of high resolution databases for terrain elevation and land use category.
%La utilización de modelos atmosféricos como WRF han sido usados intensamente en esto últimos años para predecir el comportamiento del viento en la mesoescala a través de simulaciones multiescala, sin embargo son pocos los trabajos en que se han estudiado el comportamiento hasta la microescala en simulaciones reales. Las investigaciones que analizan el comportamiento del LES para representar la CLP generalmente usan condiciones idealizadas (e.g. condiciones de borde periódica, superficies planas, imposición de gradientes de presión) lo que permite validar el acercamiento, pero a costa de perder la operatividad de trabajar en un caso real. Simular un caso real implica la utilización de bases de datos de alta resolución para la altura del terreno y la categoría de uso de suelo.

In this work, in order to obtain the best possible solution for a short-term wind forecast in PBL, the use of a 4D data assimilation system was also considered with measurements obtained in the surface proximity within the simulation time window.
%En este trabajo, con el fin de obtener la mejor solución posible para un pronóstico a corto plazo del viento en la CLP, se considera además la utilización de un sistema de asimilación de datos 4D para mediciones obtenidas en la cercanía de la superficie dentro de la ventana temporal de simulación.

Lastly, the philosophy of this work is to establish the foundations of a new method for assessing the wind resource without relying on ad hoc idealizations, but through fundamental physics and the correct implementation of state-of-the-art instrumentation. The results obtained will serve as a benchmark for future verification of new or experimental models that perform high-resolution simulations in real terrain.
%Finalmente, la filosofía de este trabajo es sentar las bases de una nueva manera de estimar el recurso eólico, sin depender de idealizaciones ad hoc, si no que a través de física fundamental y la correcta implementación de instrumentación de vanguardia. Los resultados obtenidos servirán como referencia para pruebas futuras de modelos nuevos o experimentales que realicen simulaciones a alta resolución en terreno real.

\textcolor{red}{The article is structured as follows. First, the methodology is presented, detailing the numerical model, the parameterization of the turbulence, the process of data assimilation and the experimental designs. Then, the first and second order results obtained are shown together with the corresponding analyses for the cases without and with data assimilation. Finally, the conclusions and benefits of this system are presented and general guidelines are given for future work, such as its application in complex geometry.}
%La organización de este artículo es de la siguiente forma, primero se presenta la metodología en donde se detalla el modelo numérico, la parametrización de la turbulencia, el proceso de asimilación de datos y el diseño de los experimentos. Luego, se muestran los resultados obtenidos de primer y segundo orden junto con los análisis correspondientes para los casos sin y con asimilación de datos. Finalmente se presentan las conclusiones, los beneficios de este sistema y se plantean lineamientos generales para el trabajo futuro como lo es su aplicación en terrenos con geometría compleja. 
